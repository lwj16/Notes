\documentclass[12pt, a4paper, oneside, UTF8]{ctexart}
\usepackage{amsmath, amsthm, amssymb, bm, color, framed, graphicx, hyperref, mathrsfs}
\usepackage{geometry}
\geometry{left = 2.5 cm, right = 2.5 cm, top = 2.5 cm, bottom = 2.5 cm}

\title{\textbf{作业3}\\{\small (数值算法与案例分析)}}
\author{李维杰}
\date{\today}
\linespread{1.5}
\definecolor{shadecolor}{RGB}{241, 241, 255}
\newcounter{problemname}
\newenvironment{problem}{\begin{shaded}\stepcounter{problemname}\par\noindent\textbf{题目\arabic{problemname}. }}{\end{shaded}\par}
\newenvironment{solution}{\par\noindent\textbf{解答. }}{\par}
\newenvironment{note}{\par\noindent\textbf{注记. }}{\par}

\begin{document}

\maketitle

\begin{problem}
    设$$A=\left[
        \begin{array}{cccc}	
            610 & 987 \\
            987 & 1597 
        \end{array}
    \right].$$
    考察线性方程组$Ax=b$,要求分别构造$b$及其微小扰动$\delta{b}$,使之分别满足以下两种情形:\\
    (1) $\frac{{\left\lVert{\delta{b}}\right\rVert}_{\infty}}{{\left\lVert{b}\right\rVert}_{\infty}}$极小时,$\frac{{\left\lVert{\delta{x}}\right\rVert}_{\infty}}{{\left\lVert{x}\right\rVert}_{\infty}}$极大.\\
    (2) $\frac{{\left\lVert{\delta{b}}\right\rVert}_{\infty}}{{\left\lVert{b}\right\rVert}_{\infty}}$极大时,$\frac{{\left\lVert{\delta{x}}\right\rVert}_{\infty}}{{\left\lVert{x}\right\rVert}_{\infty}}$极小.
\end{problem}

\begin{solution}
    (1) 构造
    \begin{align*}
        b = \left[
            \begin{array}{cccc}	
                1597\\
                2584
            \end{array}
        \right],
        \delta{b} = \left[
            \begin{array}{cccc}	
                -0.1\\
                0.1
            \end{array}
        \right].
    \end{align*}
    此时$Ax=b$的解向量为
    \begin{align*}
        x = \left[
            \begin{array}{cccc}	
                1\\
                1
            \end{array}
        \right],
        \delta{x} = \left[
            \begin{array}{cccc}	
                -256.4\\
                159.7
            \end{array}
        \right].
    \end{align*}
    相对误差分别为
    \begin{align*}
        \frac{{\left\lVert{\delta{b}}\right\rVert}_{\infty}}{{\left\lVert{b}\right\rVert}_{\infty}}={3.87\times{10^{-5}}},
        \frac{{\left\lVert{\delta{x}}\right\rVert}_{\infty}}{{\left\lVert{x}\right\rVert}_{\infty}}={2.56\times{10^{2}}}.
    \end{align*}
    (2) 构造
    \begin{align*}
        b = \left[
            \begin{array}{cccc}	
                -0.110\\
                -0.118
            \end{array}
        \right],
        \delta{b} = \left[
            \begin{array}{cccc}	
                159.7\\
                258.4
            \end{array}
        \right].
    \end{align*}
    此时$Ax=b$的解向量为
    \begin{align*}
        x = \left[
            \begin{array}{cccc}	
                9.886\\
                -6.110
            \end{array}
        \right],
        \delta{x} = \left[
            \begin{array}{cccc}	
                0.1\\
                0.1
            \end{array}
        \right].
    \end{align*}
    相对误差分别为
    \begin{align*}
        \frac{{\left\lVert{\delta{b}}\right\rVert}_{\infty}}{{\left\lVert{b}\right\rVert}_{\infty}}={2.19\times{10^{3}}},
        \frac{{\left\lVert{\delta{x}}\right\rVert}_{\infty}}{{\left\lVert{x}\right\rVert}_{\infty}}={1.01\times{10^{-2}}}.
    \end{align*}
\end{solution}

\begin{problem}
    设$Z\in\mathbb{C}^{{n}\times{n}}$,令$$A=\left[
        \begin{array}{cccc}	
            I_n & Z \\
            0 & I_n
        \end{array}
    \right].$$
    求$\kappa_{\mathsf{F}}(A)={\left\lVert{A}\right\rVert}_{\mathsf{F}}{\left\lVert{A^{-1}}\right\rVert}_{\mathsf{F}}$.
\end{problem}

\begin{solution}
    考虑
    \begin{align*}
        {A^*}{A}&=\left[
        \begin{array}{cccc}	
            I_n & 0 \\
            Z^* & I_n
        \end{array}
    \right]\times\left[
        \begin{array}{cccc}	
            I_n & Z \\
            0 & I_n
        \end{array}
    \right]\\
    &=\left[
        \begin{array}{cccc}	
            I_n & Z \\
            Z^* & {Z^*}{Z}+I_n
        \end{array}
    \right].
    \end{align*}
    故
    \begin{align*}
        {\left\lVert{A}\right\rVert}_{\mathsf{F}}=(tr({A^*}{A}))^{1/2}=(2n+{\left\lVert{Z}\right\rVert}_{\mathsf{F}}^2)^{1/2}.
    \end{align*}
    由于${A^{-1}}=\left[
        \begin{array}{cccc}	
            I_n & -Z \\
            0 & I_n
        \end{array}
    \right]$,则有
    \begin{align*}
        {(A^*)^{-1}}{A^{-1}}&=\left[
        \begin{array}{cccc}	
            I_n & 0 \\
            -Z^* & I_n
        \end{array}
    \right]\times\left[
        \begin{array}{cccc}	
            I_n & -Z \\
            0 & I_n
        \end{array}
    \right]\\
    &=\left[
        \begin{array}{cccc}	
            I_n & -Z \\
            -Z^* & {Z^*}{Z}+I_n
        \end{array}
    \right].
    \end{align*}
    故
    \begin{align*}
        {\left\lVert{A^{-1}}\right\rVert}_{\mathsf{F}}=(tr({(A^*)^{-1}}{A^{-1}}))^{1/2}=(2n+{\left\lVert{Z}\right\rVert}_{\mathsf{F}}^2)^{1/2}.
    \end{align*}
    于是
    \begin{align*}
        \kappa_{\mathsf{F}}(A)=2n+{\left\lVert{Z}\right\rVert}_{\mathsf{F}}^2.
    \end{align*}
\end{solution}

\begin{problem}
    可以证明进行没有选主元操作的高斯消元在解严格对角占优线性方程组时具有数值稳定性,即此时增长因子有界.求这里增长因子的上界.\\
    \textbf{注:} 增长因子定义为$\rho=\max\limits_{i,j,k}{\frac{|a_{i,j}^{(k)}|}{{\left\lVert{A}\right\rVert}_{\infty}}}$.
\end{problem}

\begin{solution}
    设经过$k$步消元后,$A$阵变为${A^{(k+1)}}={L_k}{A}$.考察$A^{(k+1)}$阵中的某个元素的模数
    \begin{align*}
        |a_{ij}^{(k+1)}|=|{a_{ij}^{(k)}}+\frac{a_{ik}^{(k)}}{a_{kk}^{(k)}}{a_{kj}^{(k)}}|\leq{|{a_{ij}^{(k)}}|+|{\frac{a_{kj}^{(k)}}{a_{kk}^{(k)}}}||{a_{ik}^{(k)}}|}.
    \end{align*}
    结合上次作业中证明得到的结论:严格对角占优的矩阵在高斯消元过程中得到的中间矩阵依然严格对角占
    优.可知
    \begin{align*}
        \sum\limits_{j=k+1}^{n}{|a_{ij}^{(k+1)}|}&\leq\sum\limits_{j=k+1}^{n}{|{a_{ij}^{(k)}}|}+|{a_{ik}^{(k)}}|\sum\limits_{j=k+1}^{n}{|\frac{a_{kj}^{(k)}}{a_{kk}^{(k)}}|}\\
        &\leq\sum\limits_{j=k+1}^{n}{|{a_{ij}^{(k)}}|}+|{a_{ik}^{(k)}}|\\
        &=\sum\limits_{j=k}^{n}{|{a_{ij}^{(k)}}|}.
    \end{align*}
    依此类推,得
    \begin{align*}
        \sum\limits_{j=k+1}^{n}{|a_{ij}^{(k+1)}|}\leq\sum\limits_{j=1}^{n}{|a_{ij}^{(1)}|}.
    \end{align*}
    因此
    \begin{align*}
        \max\limits_{i,j,k}{|{a_{ij}^{(k)}}|}&\leq{\max\limits_{i,k}{\sum\limits_{j=k}^{n}{|{a_{ij}^{(k)}}|}}}\\
        &\leq\max\limits_{i}{\sum\limits_{j=1}^{n}{|{a_{ij}^{(1)}}|}}\\
        &={\left\lVert{A}\right\rVert}_{\infty}.
    \end{align*}
    即得$\rho\leq{1}$.且当$A=I$时恰有$\rho=1$,故上界可以取到.
\end{solution}
\newpage
\begin{problem}
    可以证明进行带列选主元的高斯消元在解非奇异三对角线性方程组时具有数值稳定性,即此时增长因子有界.求这里增长因子的上界.\\
    \textbf{注:} 增长因子定义为$\rho=\max\limits_{i,j,k}{\frac{|a_{i,j}^{(k)}|}{{\left\lVert{A}\right\rVert}_{\infty}}}$.
\end{problem}

\begin{solution}
    设$A$阵经列主元行变换得到的矩阵为$B^{(1)}$阵,即$B^{(1)}=PA$.考虑$B^{(1)}$阵中任意元素的模数,由于$B^{(k)}$中列主元占优,故
    \begin{align*}
        |{b_{ij}^{(k+1)}}|=|{b_{ij}^{(k)}+\frac{b_{ik}^{(k)}}{b_{kk}^{(k)}}{b_{kj}^{(k)}}}|\leq{|{b_{ij}^{(k)}}|+|\frac{b_{ik}^{(k)}}{b_{kk}^{(k)}}||{b_{kj}^{(k)}}|},&&i>k+1.
    \end{align*}
    由引理可知只需考虑$j=k+1,j=k+2,j=k+3$三种情况.\\
    当$j=k+3$时,
    \begin{align*}
        |{b_{i(k+3)}^{(k+1)}}|=|{b_{i(k+3)}^{(1)}}|.
    \end{align*}
    当$j=k+2$时,
    \begin{align*}
        |{b_{i(k+2)}^{(k+1)}}|&\leq{|{b_{i(k+2)}^{(k)}}|+|\frac{b_{ik}^{(k)}}{b_{kk}^{(k)}}||{b_{k(k+2)}^{(k)}}|}\\
        &={|{b_{i(k+2)}^{(1)}}|+|\frac{b_{ik}^{(k)}}{b_{kk}^{(k)}}||{b_{k(k+2)}^{(1)}}|}.
    \end{align*}
    当$j=k+1$时,
    由引理知${b_{(k-1)(k-1)}^{(k-1)}},{b_{i(k-1)}^{(k-1)}},{b_{k(k-1)}^{(k-1)}}$中应当仅有两个不为零,且主元${b_{(k-1)(k-1)}^{(k-1)}}$必定不为零,故
    \begin{align*}
        \left\{
            \begin{array}{ll}
                {b_{i(k-1)}^{(k-1)}}=0 \\
                {0}<\frac{b_{k(k-1)}^{(k-1)}}{b_{(k-1)(k-1)}^{(k-1)}}<1
            \end{array}
        \right.
        \text{或}
        \left\{
            \begin{array}{ll}
                {b_{k(k-1)}^{(k-1)}}=0 \\
                {0}<\frac{b_{i(k-1)}^{(k-1)}}{b_{(k-1)(k-1)}^{(k-1)}}<1
            \end{array}
        \right.
    \end{align*}
    于是得到
    \begin{align*}
        |{b_{i(k+1)}^{(k+1)}}|&\leq{|{b_{i(k+1)}^{(k)}}|+|{b_{k(k+1)}^{(k)}}|}\\
        &\leq{|{b_{i(k+1)}^{(1)}}|+|\frac{b_{i(k-1)}^{(k-1)}}{b_{(k-1)(k-1)}^{(k-1)}}||{b_{(k-1)(k+1)}^{(1)}}|+|{b_{k(k+1)}^{(1)}}|+|\frac{b_{k(k-1)}^{(k-1)}}{b_{(k-1)(k-1)}^{(k-1)}}||{b_{(k-1)(k+1)}^{(1)}}|}\\
        &\leq{3\max\limits_{i,j}{|b_{ij}^{(1)}|}}
    \end{align*}
    由于$A$阵中各元素地位等价,故在理想状态下,可令${\left\lVert{A}\right\rVert}_{\infty}={3\max\limits_{i,j}{|a_{ij}^{(1)}|}}$.因此
    \begin{align*}
        \max\limits_{i,j,k}{|{a_{ij}^{(k)}}|}&=\max\limits_{i,j,k}{|{b_{ij}^{(k)}}|}\\
        &\leq{3\max\limits_{i,j}{|b_{ij}^{(1)}|}}\\
        &={3\max\limits_{i,j}{|a_{ij}^{(1)}|}}\\
        &={\left\lVert{A}\right\rVert}_{\infty}.
    \end{align*}
    即得$\rho\leq{1}$.且当$A=I$时恰有$\rho=1$,故上界可以取到.\\\\
    \textbf{引理.}
    设$A$阵经列选主元行变换得到$B$阵.若$b_{ik}^{(k)}\neq{0}$,则$\forall{t\geq{k+3}}$,均有$b_{it}^{(k)}=0$.\\
    \textbf{证明.} 
    当$n=3$时,结论显然成立.\\
    假设当$n=m$时,结论成立.现考虑$n=m+1$的情况.\\
    对于$A$阵,仅有$a_{11}^{(1)}\neq{0},a_{21}^{(1)}\neq{0}$.则$B$阵的第一行或为$A$阵的第一行(非零列数为2),或为$A$阵的第二行(非零列数为3).于是在第一步高斯消元中,仅有一行的至多前三列被更新.\\
    因此,对于第一步高斯消元后$B^{(2)}$阵,其右下角的${m}\times{m}$阵与原先的$B$阵拥有同样的性质,即该子阵的第一行或为非零列数为2的一行,或为非零列数为3的一行.由假设可知该子阵在后续的高斯消元中保持结论性质不变.\\
    综上,结论得证.
\end{solution}
\newpage
\begin{problem}
    分别利用选主元和不选主元的高斯消元法解以下两个线性方程组,并把解向量与精确解作比较,讨论解法的数值稳定性.\\
    (1)
    \begin{align*}
        \left[
            \begin{array}{ccccccc}	
                8 & 1\\	
                6 & 8 & 1 \\
                  & 6 & 8 & 1 \\
                  &   & \ddots & \ddots & \ddots \\
                  &   &   & 6 & 8 & 1 \\
                  &   &   &   & 6 & 8 & 1 \\
                  &   &   &   &   & 6 & 8 
            \end{array}
        \right]
        \left[
            \begin{array}{c}
                x_{1} \\
                x_{2} \\
                x_{3} \\
                \vdots \\
                x_{98} \\
                x_{99} \\
                x_{100}
            \end{array}
        \right]
        =
        \left[
            \begin{array}{c}
                9 \\
                15 \\
                15 \\
                \vdots \\
                15 \\
                15 \\
                14
            \end{array}
        \right].
    \end{align*}
    (2)
    \begin{align*}
        \left[
            \begin{array}{ccccccc}	
                6 & 1\\	
                8 & 6 & 1 \\
                  & 8 & 6 & 1 \\
                  &   & \ddots & \ddots & \ddots \\
                  &   &   & 8 & 6 & 1 \\
                  &   &   &   & 8 & 6 & 1 \\
                  &   &   &   &   & 8 & 6 
            \end{array}
        \right]
        \left[
            \begin{array}{c}
                x_{1} \\
                x_{2} \\
                x_{3} \\
                \vdots \\
                x_{98} \\
                x_{99} \\
                x_{100}
            \end{array}
        \right]
        =
        \left[
            \begin{array}{c}
                7 \\
                15 \\
                15 \\
                \vdots \\
                15 \\
                15 \\
                14
            \end{array}
        \right].
    \end{align*}
\end{problem}

\begin{solution}
    (1) 选主元高斯消元:$$\frac{{\left\lVert{\delta{x}}\right\rVert}_{\infty}}{{\left\lVert{x}\right\rVert}_{\infty}}=2.2204\times{10^{-16}}.$$
    不选主元高斯消元:$$\frac{{\left\lVert{\delta{x}}\right\rVert}_{\infty}}{{\left\lVert{x}\right\rVert}_{\infty}}=2.2204\times{10^{-16}}.$$
    (2) 选主元高斯消元:$$\frac{{\left\lVert{\delta{x}}\right\rVert}_{\infty}}{{\left\lVert{x}\right\rVert}_{\infty}}=1.8333\times{10^{-1}}.$$
    不选主元高斯消元:$$\frac{{\left\lVert{\delta{x}}\right\rVert}_{\infty}}{{\left\lVert{x}\right\rVert}_{\infty}}=3.5182\times{10^{13}}.$$
    综上,利用高斯消元法解非严格对角占优的线性方程组时,选主元法的数值稳定性远优于不选主元法.(本题的详细代码见codes文件夹内的Problem5.m)
\end{solution}

\end{document}