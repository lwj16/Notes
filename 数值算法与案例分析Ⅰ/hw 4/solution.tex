\documentclass[12pt, a4paper, oneside, UTF8]{ctexart}
\usepackage{amsmath, amsthm, amssymb, bm, color, framed, graphicx, hyperref, mathrsfs}
\usepackage{geometry}
\geometry{left = 2.5 cm, right = 2.5 cm, top = 2.5 cm, bottom = 2.5 cm}

\title{\textbf{作业4}\\{\small (数值算法与案例分析)}}
\author{李维杰}
\date{\today}
\linespread{1.5}
\definecolor{shadecolor}{RGB}{241, 241, 255}
\newcounter{problemname}
\newenvironment{problem}{\begin{shaded}\stepcounter{problemname}\par\noindent\textbf{题目\arabic{problemname}. }}{\end{shaded}\par}
\newenvironment{solution}{\par\noindent\textbf{解答. }}{\par}
\newenvironment{note}{\par\noindent\textbf{注记. }}{\par}

\begin{document}

\maketitle

\begin{problem}
    对于$x,y\in{\mathbb{C}}^n$,我们知道${\left\lVert{x}\right\rVert}_{2}={\left\lVert{y}\right\rVert}_{\mathsf{2}}>0$无法保证Householder矩阵的存在.请给出Householder矩阵存在的充要条件,并证明.
\end{problem}

\begin{solution}
    Householder矩阵存在的充要条件是${\left\lVert{x}\right\rVert}_{2}={\left\lVert{y}\right\rVert}_{\mathsf{2}}>0$且${x}\neq{y}$.下面对此加以证明:
    \par\textbf{必要性.} 当$H(\omega)=I-\frac{2\omega{\omega}^*}{{\omega}^*\omega}$存在时,即有$${\omega}^*\omega={\left\lVert{\omega}\right\rVert}_{2}^{2}\neq{0}.$$
    其中$\omega=x-y$,故必有${x}\neq{y}.$另外,由于
    \begin{align*}
        H(\omega)^{*}H(\omega)&=(I-\frac{2}{{\omega}^{*}{\omega}}({\omega}{\omega}^{*})^{*})H(\omega)\\
        &=(I-\frac{2}{{\omega}^{*}{\omega}}{\omega}{\omega}^{*})^2\\
        &=I-\frac{4}{{\omega}^{*}{\omega}}{\omega}{\omega}^{*}+\frac{4}{({\omega}^{*}{\omega})^2}{\omega}({\omega}^{*}{\omega}){\omega}^{*}\\
        &=I.
    \end{align*}
    因此
    \begin{align*}
        {\left\lVert{y}\right\rVert}_{2}^{2}&={\left\lVert{H(\omega)x}\right\rVert}_{2}^{2}\\
        &=(H(\omega)x)^{*}(H(\omega)x)\\
        &=x^{*}H(\omega)^{*}H(\omega)x\\
        &=x^{*}x={\left\lVert{x}\right\rVert}_{2}^{2}.
    \end{align*}
    故有${\left\lVert{x}\right\rVert}_{2}={\left\lVert{y}\right\rVert}_{2}>0.$
    \par \textbf{充分性.} 当${x}\neq{y}$时,必有${\omega}^*{\omega}={\left\lVert{\omega}\right\rVert}_{2}^{2}\neq{0}.$,于是Householder矩阵存在.

\end{solution}

\begin{problem}
    说明在复矩阵的Householder三角化算法中如何避免有效数字的抵消.
\end{problem}

\begin{solution}
    对于Householder矩阵$H(\omega)=I-\frac{2\omega{\omega}^*}{{\omega}^*\omega}$,其中
    $$\omega=x-y=x-{\left\lVert{x}\right\rVert}_{2}e_1.$$
    当$x$很接近基向量$e_1$时,会产生有效数字的抵消.设$x=(x_k,x_{k+1},...,x_n)^T$,则在$x$很接近$e_1$时,可作恒等变换
    \begin{align*}
        \omega=x-{\left\lVert{x}\right\rVert}_{2}&=\frac{x^2-{\left\lVert{x}\right\rVert}_{2}^{2}}{x+{\left\lVert{x}\right\rVert}_{2}}=-\frac{\sum_{i=k+1}^{n}{x_i^2}}{x+{\left\lVert{x}\right\rVert}_{2}}.
    \end{align*}
    即可避免相近数相减造成的有效数字抵消问题.
\end{solution}

\begin{problem}
    分别用以下方法编写计算复矩阵$A\in{\mathbb{C}}^{{n}\times{n}}$的$QR$分解的程序.\\
    (1) Cholesky QR (例如:通过$A^*A$的Cholesky分解).\\
    (2) Householder triangularization.\\
    并分别以良态和病态的例子表现算法的正交性损失(即$|{Q^*Q-I_n}|$)
\end{problem}

\begin{solution}
    (1) 由$$LL^*=A^*A=R^*Q^*QR=R^*R,$$得$R=L^*$.且有${R^*Q^*=A^*}\Rightarrow{LQ=A^*}$.从而可通过Cholesky分解求出$A$的$QR$分解.\\
    (2) 利用Householder变换迭代即可求出$A$的$QR$分解.\\
    
    在本题中,选用良态矩阵为($\kappa(A_1)=6.494$):
    \begin{align*}
        A_1 =
        \left[
            \begin{array}{ccccc}	
                2 + 3i & 1 & 0 & 0 & 0 \\
                1 & 4 - i & 1 & 0 & 0 \\
                0 & 1 & 3 + 2i & 1 & 0 \\
                0 & 0 & 1 & 5 & 1 - i \\
                0 & 0 & 0 & 1 & 6 + i
            \end{array}
        \right],
    \end{align*}
    此时,算法(1)的正交性损失为$1.624\times{10^{-1}}$,算法(2)的正交性损失为$1.463\times{10^{-15}}$.\\
    
    选用病态矩阵为($\kappa(A_2)=4.540\times{10^{3}}$):
    \begin{align*}
        A_2 =
        \left[
            \begin{array}{ccccc}	
                1 & 0.999 + 0.001i & 0.999 + 0.002i & 0.999 + 0.003i & 0.999 + 0.004i \\
                0.999 + 0.001i & 1 & 0.999 + 0.005i & 0.999 + 0.006i & 0.999 + 0.007i \\
                0.999 + 0.002i & 0.999 + 0.005i & 1 & 0.999 + 0.008i & 0.999 + 0.009i \\
                0.999 + 0.003i & 0.999 + 0.006i & 0.999 + 0.008i & 1 & 0.999 + 0.010i \\
                0.999 + 0.004i & 0.999 + 0.007i & 0.999 + 0.009i & 0.999 + 0.010i & 1
            \end{array}
        \right],
    \end{align*}
    此时,算法(1)的正交性损失为$4.160$,算法(2)的正交性损失为$1.041\times{10^{-15}}$.
\end{solution}

\begin{note}
    本题表明HouseholderQR分解具有相当优秀的数值稳定性.
\end{note}

\begin{problem}
    设
    \begin{align*}
        A=
        \left[
            \begin{array}{cccccc}	
                {\alpha}_1 & {\rho}_2 & {\rho}_3 & \cdots & \cdots & {\rho}_n \\
                {\beta}_2 & {\alpha}_2 & 0 & \cdots & \cdots & 0 \\
                {\beta}_3 & 0 & {\alpha}_3 & \ddots &   & \vdots \\
                \vdots & \vdots & \ddots & \ddots & \ddots & \vdots \\
                \vdots & \vdots &   & \ddots & {\alpha}_{n-1} & 0 \\
                {\beta}_n & 0 & \cdots & \cdots & 0 & {\alpha}_n
            \end{array}
        \right],
    \end{align*}
    设计一个高效的算法,计算$A$的$QR$分解.
\end{problem}

\begin{solution}
    由于$A$阵是一个稀疏矩阵,故采用Givens变换进行$QR$分解更优.具体操作如下:\\
    设
    \begin{align*}
        T_{1j}=
        \left[
            \begin{array}{cccccccc}	
                c_j &   &   &   & s_j\\
                  & 1 \\
                  &   & \ddots \\
                  &   &   & 1 \\
                -s_j &   &   &   & c_j \\
                  &   &   &   &   & 1 \\
                  &   &   &   &   &   & \ddots \\
                  &   &   &   &   &   &    &  1
            \end{array}
        \right],
    \end{align*}
    其中非$0/1$元素的坐标为$(1,1),(1,j),(j,1),(j,j)$.且
    \begin{align*}
        \left\{
            \begin{array}{ll}
                c_j=\frac{{\beta}_{j}}{\sqrt{{\alpha}_{1}^{2}+{\beta}_{j}^{2}}} \\
                s_j=\frac{{\alpha}_{1}}{\sqrt{{\alpha}_{1}^{2}+{\beta}_{j}^{2}}}
            \end{array}.
        \right.
    \end{align*}
    则$T_{1j}$是酉阵,于是
    \begin{align*}
        Q=\prod_{j=2}^{n}{T_{1j}^{-1}},
        R=\prod_{j=2}^{n}{T_{1j}}{A}=
        \left[
            \begin{array}{cccccc}	
                {\alpha}_{1}^{'} & {\rho}_2 & {\rho}_3 & \cdots & \cdots & {\rho}_n \\
                0 & {\alpha}_2 & 0 & \cdots & \cdots & 0 \\
                0 & 0 & {\alpha}_3 & \ddots &   & \vdots \\
                \vdots & \vdots & \ddots & \ddots & \ddots & \vdots \\
                \vdots & \vdots &   & \ddots & {\alpha}_{n-1} & 0 \\
                0 & 0 & \cdots & \cdots & 0 & {\alpha}_n
            \end{array}
        \right].
    \end{align*}
    其中${\alpha}_{1}^{'}=\sqrt{{\alpha}_{1}^{2}+\sum\limits_{j=2}^{n}{{\beta}_{j}^{2}}}$.\\
    上述算法中,计算$T_{1j}$的效率是$O(1)$的,计算$R$阵的效率是$O(n)$的,
    计算$Q$阵的效率是$\sum\limits_{i=2}^{n}{O(n)}=O(n^2)$的,故算法的总时间复杂度是$O(n^2)$的.
\end{solution}

\begin{problem}
    设$Q\in{\mathbb{R}}^{{n}\times{n}}$是一个正交阵.证明:\\
    (1) $Q$可以被分解为有限个Householder reflections $H$的乘积.\\
    (2) 若$\text{det}(Q)=1$,则$Q$可以被分解为有限个Givens rotations $T$的乘积.
\end{problem}

\begin{solution}
    (1) 定义$col_i(A)$表示矩阵$A$的第$i$列向量.由于$Q$是正交阵,故$\{col_1(Q),col_2(Q),\cdots,col_n(Q)\}$是一组正交基.设
    \begin{align*}
        {\omega}_{i}=col_i(I_n)-\frac{col_i(Q)}{{\left\lVert{col_i(Q)}\right\rVert}_{2}},
    \end{align*}
    则$H({\omega}_{i})$将$I_n$阵的第$i$行镜射到$Q$阵的第$i$列向量方向上,因此
    \begin{align*}
        Q=I_n\times\prod\limits_{i=1}^{n}{{\left\lVert{col_i(Q)}\right\rVert}_{2}H({\omega}_{i})}=\prod\limits_{i=1}^{n}{H({\omega}_{i})}.
    \end{align*}
    (2) 设Givens rotations $T_{ij}$中
    \begin{align*}
        \left\{
            \begin{array}{ll}
                c=\frac{{q}_{ji}}{\sqrt{{q}_{ii}^{2}+{q}_{ji}^{2}}} \\
                s=-\frac{{q}_{ii}}{\sqrt{{q}_{ii}^{2}+{q}_{ji}^{2}}}
            \end{array}.
        \right.
    \end{align*}
    则$T_{ij}$相当于把$I_n$阵中的一组二维向量旋转至$Q$阵中对应位置,即
    \begin{align*}
        Q=I_n\times\prod\limits_{i=1}^{n}{{\left\lVert{col_i(Q)}\right\rVert}_{2}\prod\limits_{j=1,{j}\neq{i}}^{n}{T_{ij}}}=\prod\limits_{i=1}^{n}{\prod\limits_{j=1,{j}\neq{i}}^{n}{T_{ij}}}.
    \end{align*}
    又由于$\det(T_{ij})=\det(\left[
        \begin{array}{cc}	
            \cos{\theta} & \sin{\theta} \\
            -\sin{\theta} & \cos{\theta}
        \end{array}
    \right])=1$.故要求上式成立时,应当满足前提$$\det(Q)=\det(\prod\limits_{i=1}^{n}{\prod\limits_{j=1,{j}\neq{i}}^{n}{T_{ij}}})=\prod\limits_{i=1}^{n}{\prod\limits_{j=1,{j}\neq{i}}^{n}{\det(T_{ij})}}=1.$$
\end{solution}
\end{document}